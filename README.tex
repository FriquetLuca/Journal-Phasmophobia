\documentclass[12pt, letterpaper]{article}
\usepackage[utf8]{inputenc}
\usepackage{amsthm}
\usepackage{amssymb}
\newtheorem{theorem}{Theorem}

\begin{document}
    \author{Luca Friquet \thanks{inspired by Insym}}

    \section{Ghosts}

    \begin{theorem}[Total number of ghosts]
        There can only be 35 ghosts in the game as of right now.
    \end{theorem}
    \begin{proof}
        Let's suppose we don't know the total number of all ghosts in the game.\\
        There's 7 evidences: "EMF 5", "Fingerprints", "Ghost Writing", "Freezing Temperature", "Spirit Box", "DOTS" and "Ghost Orbs".\\
        We know for a fact that all ghosts must have 3 distinct evidences out of those 7 (let's not consider the mimic who's a special case since he can have 4 evidences).\\
        One would assume that there's ${7 \choose 3} = \frac{7!}{3!(7-3)!} = 35$ ghosts (there's in fact only 24 ghosts).
        Based on that, we know for a fact that there can be a total of $35$ ghosts at most.
    \end{proof}

    \begin{theorem}[One evidence total]
        There can be at most 15 ghosts if we have only one evidence.
    \end{theorem}
    \begin{proof}
        Since we assume we have a proof of one evidence, there must be ${6 \choose 3}=20$ ghosts that doesn't possess that said evidence.\\
        Based on that fact, we know that we have a total of $15$ ghosts to deal with (since ${7 \choose 3} - {6 \choose 3} = {6 \choose 2} = 15$).
    \end{proof}

    \begin{theorem}[Two evidence total]
        There can be at most 5 ghosts if we have two evidences.
    \end{theorem}
    \begin{proof}
        Since we assume we have a proof of two evidences, there must be ${5 \choose 1}=5$ ghosts that matched those evidences.
    \end{proof}

    \begin{theorem}[Evidence combination]
        There can be at most 21 ghosts for two pairs of evidences.
    \end{theorem}
    \begin{proof}
        Since we assume we have two pairs of evidences for the 7 possibles evidences in the game, there must be ${7 \choose 2}=21$ ghosts that matched those evidences.
    \end{proof}

    \section{Nightmare Mode}

    \begin{theorem}[Nightmare Ghost Elimination]
        If there's one evidence and there's a proof of a lack of two evidence, then the ghost that use those two lacking evidence can be ruled out as a result.
    \end{theorem}
    \begin{proof}
        We assume that two evidences don't exist, this means we're looking after a ghost that have only two evidences since only one evidence is hidden.\\
        That's contradicting the fact that there's only one hidden evidence by the rules of nightmare mode.\\
        Since there's an unique ghost that need both of theses evidences, it can't be the one we're looking for.
    \end{proof}
\end{document}